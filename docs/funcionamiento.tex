\documentclass[12pt, a4paper]{article}

\usepackage[utf8]{inputenc}
\usepackage[spanish, es-tabla]{babel}

\usepackage[a4paper, margin=2.5cm]{geometry}
\linespread{1.2}

\usepackage{xcolor}
\usepackage{graphicx}
\usepackage{hyperref}
\usepackage{listings}
\usepackage{booktabs}

\hypersetup{
    colorlinks=true,
    linkcolor=blue!70!black,
    urlcolor=cyan!70!black,
}

\renewcommand{\lstlistingname}{Código}

\lstdefinestyle{mystyle}{
    backgroundcolor=\color{black!5},
    basicstyle=\ttfamily\footnotesize,
    breaklines=true,
    frame=single,
    framerule=0.5pt,
    rulecolor=\color{black!20},
    numbers=left,
    numberstyle=\tiny\color{gray},
    numbersep=5pt,
    captionpos=b,
}
\lstset{style=mystyle}

\begin{document}

\begin{titlepage}
    \centering
    \includegraphics[width=0.5\textwidth]{images/inngest.png}
    
    \vspace{3cm}
    
    \Huge\bfseries
    Documentación Técnica: \\
    Bot de Telegram con Inngest
    
    \vfill
    
    \Large
    \textbf{Autor:} Pedro José Meixús Belsol \\
    \vspace{0.5cm}
    \today
\end{titlepage}

\tableofcontents
\newpage

\section{Introducción}
En esta documentación se detalla el proceso de creación de un bot de Telegram y funcionamiento con nuestra app con Inngest. La app permite al usuario escribir un mensaje y que el bot les deuvelva el mensaje.

\section{Creación del Bot de Telegram}

Primero, creamos un bot de Telegram utilizando el BotFather, obteniendo un token de autenticación necesario para interactuar con la API de Telegram.

\vspace{0.5cm}

\begin{center}
    \includegraphics[width=0.8\textwidth]{images/creacion_bot.jpg}
\end{center}

\vspace{0.5cm}

Y ahora iniciamos la conversación con nuestro bot.

\vspace{0.5cm}

\begin{center}
    \includegraphics[width=0.8\textwidth]{images/iniciar_bot.jpg}
\end{center}

\vspace{0.5cm}

\section{Funcionamiento}

Vamos a explicar el funcionamiento de la aplicación. Primero lanzaremos la API con el siguiente comando:

\vspace{0.5cm}

\begin{center}
    \includegraphics[width=0.8\textwidth]{images/iniciar_api.png}
\end{center}

\vspace{0.5cm}

Al iniciarse correctamente deberiamos ver ese mensaje de que está escuchando en el puerto 3000.

Ahora iniciamos el Inngest-cli con el siguiente comando:

\vspace{0.5cm}

\begin{center}
    \includegraphics[width=0.8\textwidth]{images/iniciar_inngest.png}
\end{center}

\vspace{0.5cm}

Si todo ha ido bien, al entrar en 'localhost:3000' deberiamos ver el input para escribir el mensaje que queremos que nos devuelva el bot.

\vspace{0.5cm}

\begin{center}
    \includegraphics[width=1\textwidth]{images/input_mensaje.png}
\end{center}

\vspace{0.5cm}

Al escribir el mensaje y darle a enviar, el bot de Telegram nos devolverá el mismo mensaje.

\vspace{0.5cm}

\begin{center}
    \includegraphics[width=1\textwidth]{images/respuesta_bot.jpg}
\end{center}

\vspace{0.5cm}

Con esto hemos comprobado que el bot de Telegram funciona correctamente con nuestra app con Inngest.

\end{document}